\documentclass{article}
\usepackage{amsmath,mathtools}
\usepackage{graphicx}
\usepackage{mathrsfs}
\usepackage{braket}

\title{Symmetric Tensor Index Mappings Between Upper Triangular Multidimensional and Flattened (Raveled) Indices}
\date{11-30-2020}
\author{Adam Abbott}
\begin{document}
\maketitle

\section{Introduction}
Given a symmetric $n \times n$ symmetric matrix,

\begin{equation*}
\begin{pmatrix}
a_{1,1} & a_{1,2} & a_{1,3} & \cdots & a_{1,n} \\
a_{1,2} & a_{2,2} & a_{2,3} & \cdots & a_{2,n} \\
a_{1,3} & a_{2,3} & a_{3,3} & \cdots & a_{3,n} \\
\vdots  & \vdots  & \vdots  & \ddots & \vdots  \\
a_{1,n} & a_{2,n}& a_{3,n}  & \cdots & a_{n,n} 
\end{pmatrix}
\end{equation*}

A set of unique elements can be obtained by collecting only the upper triangle or lower triangle into a vector, including the diagonal. 
This enables efficient storage in a computer program.

\begin{equation*}
\left[ a_{1,1}, a_{1,2}, a_{1,3} , \cdots , a_{1,n} , a_{2,2} , a_{2,3} , \cdots , a_{2,n} , a_{3,3} , \cdots , a_{3,n} , \cdots , a_{n,n} \right]
\end{equation*}

Looking at a more concrete example of a $6 \times 6$ matrix, and filling the upper triangle row-wise with integers starting from 0, we are able to derive 
a mapping from a \textbf{multidimensional upper triangle index}  $[i,j]$ for $i \leq j$ to the \textbf{flattened upper triangle index}, which we will call $z$:

\begin{equation*}
\begin{pmatrix}
0 & 1 & 2 & 3 & 4 & 5 \\
  & 6 & 7 & 8 & 9 & 10 \\
  &   & 11 & 12 & 13 & 14 \\
  &   &    & 15 & 16 & 17 \\
  &   &    &    & 18 & 19 \\
  &   &    &    &    & 20 \\
\end{pmatrix}
\end{equation*}

For example, using zero-based indexing, if we index this matrix with the pair $[2,4]$ we obtain 16.
Thus, if we had \textit{any} $6 \times 6$ symmetric matrix, we would know how to build the flattened vector of unique elements,
for example we would know to take the element at $[2,4]$ and place it at index $16$ in our vector, as informed by our above mapping.
In reverse order, if we only had the vector of unique elements of the upper triangle of a symmetric matrix and we wished to construct the full matrix,
we would know that the element at index $16$ in this vector should be placed at position $[2,4]$.
This reverse mapping can also be achieved with lookup-array procedure, but in this case, we wish to map 16 to $[2,4]$, or generally any flattened index $z$ to the multidimensional pair $[i,j]$ such that $i \leq j$.
The reverse mapping array can be obtained by taking all combinations with repitition of size two from a list of  6 objects, which are sorted such that each
row is elementwise $\leq$ then the next row.

\begin{equation*}
\begin{pmatrix}
0& 0\\ 
0& 1\\
0& 2\\
0& 3\\
0& 4\\
0& 5\\
1& 1\\
1& 2\\
1& 3\\
1& 4\\
1& 5\\
2& 2\\
2& 3\\
2& 4\\
2& 5\\
3& 3\\
3& 4\\
3& 5\\
4& 4\\
4& 5\\
5& 5\\
\end{pmatrix}
\end{equation*}

In the general case of an $n \times n$ symmetric matrix, however, we would ideally not have to construct any arrays for mapping between the indices. 
Ideally, we would have an equation which given $[i,j]$ gives the flattened index $z$, and another equation for vice versa.
TODO add equations here.

However, it is trickier to define these equations in the general case of mapping any rank $r$ tensor's upper triangle multidimensional index to the flattened index.
We will derive these relations in the next section.

\section{Generalized Index Mappings for Upper Triangles of Symmetric Tensors}
As has been laid out beautifully elsewhere \footnote{Components of totally symmetric and anti-symmetirc tensors, http://www.physics.mcgill.ca/~yangob/symmetric\%20tensor.pdf},
the number of elements in the flattened upper triangle of a rank $r$ tensor with dimension size $n$, which we will call $\Delta$, is 

\begin{equation*}
\Delta = 
\begin{pmatrix}
n + r - 1 \\
r \\
\end{pmatrix}
= \frac{(n + r - 1)!}{r!(n-1)!} = \frac{(n + r - 1)(n + r - 2) \cdots (n + 1) (n)}{r!}
\end{equation*}

We are now ready to define functions which take in a sorted multidimensional index ($i \leq j \leq k ...$) and return a flattened upper triangle index, for any rank symmetric tensor.
These functions we wil denote as $Z_r(...)$ where $r$ is the rank of the symmetric tensor the equation applies to.
As it turns out, each subsequent rank equation is most concisely defined in terms of the equations for previous ranks, so we choose to define them recursively here.

For a particular dimension size $n$, we first denote a series of variables $\Delta_0,\Delta_1,\Delta_2, \Delta_3, ... \Delta_r, $ which represent the number of elements 
in the upper triangle of a rank $r$ symmetric tensor, so that

\begin{align*}
\Delta_0 &= 1 \\
\Delta_1 &= n \\
\Delta_2 &= \frac{n(n+1)}{2} \\
\Delta_3 &= \frac{n(n+1)(n+2)}{6} \\
\cdots   & \\
\Delta_r &= \frac{n(n+1)(n+2)\cdots (n + (r-1)}{r!} \\
\end{align*}

Now for every rank, we can write down the following functions.
Note that for the sake of a clear and clean pattern, we write down many redundant or non necessary symbols, such as the passing of argument $n$ to functions that have no dependence on it.

\begin{align*}
Z_0(n)& = 0 \\
Z_1(i,n)& = Z_0(n-i) + \sum_{a=1}^{i} \Delta_0  \\
Z_2(i,j,n)& = Z_1(j-i,n-i) + \sum_{a=1}^{i} \Delta_1 - \sum_{b=1}^{a-1} \Delta_0  \\
Z_3(i,j,k,n)& = Z_2(j-i,k-i,n-i) + \sum_{a=1}^{i} \Delta_2 - \sum_{b=1}^{a-1} \Delta_1 + \sum_{c=1}^{b-1} \Delta_0  \\
Z_4(i,j,k,l,n)& = Z_3(j-i,k-i,l-i,n-i) + \sum_{a=1}^{i} \Delta_3 - \sum_{b=1}^{a-1} \Delta_2 + \sum_{c=1}^{b-1} \Delta_1  - \sum_{d=1}^{c-1} \Delta_0  \\
\end{align*}

It is thus easy to write down the relation for any symmetric tensor and simplify. 
Speaking of which, these equations can be simplified tremendously, however the simplifications will spoil the clear mathematical pattern.

\end{document}
